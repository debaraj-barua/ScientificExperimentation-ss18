				\documentclass[10pt,a4paper]{article}
				\usepackage[english]{babel}
				\usepackage[utf8]{inputenc}
				\usepackage{fancyhdr}
				\usepackage{hyperref}
				\usepackage{graphicx}
				\usepackage{subcaption}
				\usepackage{caption}
				\usepackage{cite}
				\usepackage{booktabs}
				\usepackage{wrapfig}
				\usepackage{float}
				\usepackage{amsmath}
				\usepackage{xcolor}
				\usepackage{listings}
				\lstset{
					basicstyle=\ttfamily,
					columns=fullflexible,
					frame=single,
					breaklines=true,
					%postbreak=\mbox{\textcolor{red}{$\hookrightarrow$}\space},
				}
				
				\restylefloat{table}
			
				\pagestyle{fancy}
				\fancyhf{}
				\rhead{\today}
				\lhead{Assignment 05}
				\rfoot{Page \thepage}
			
				\begin{document}
				\begin{titlepage}
				\centering
			
					{\scshape\LARGE Scientific Experimentation and Evaluation\par}
			
					{\scshape\Large Assignment: 05\par}
			
					\vfill
			
					\vfill
					{\Large\itshape Anees Khan (9030423)
						\\Debaraj Barua (9030412)\\
						Md Zahiduzzaman (9030432)
						\par}
					\vfill
			
					{\large 14-June-2018\par}
				\end{titlepage}
				\tableofcontents
				\listoffigures	
				\listoftables
				\newpage
				\section{Relevant Aspects of Experiment}
				\subsection{Apparatus}
					\subsubsection{Hardware}
						\begin{itemize}
							\item KUKA youBot arm.
							\item Objects of three different sizes and weights, with ArUco markers attached to the top.
							\item Camera (Microsoft LifeCam).
							\item Two computers, one to run the robot and other for data gathering from the camera.
							\item A fixed container or marker on the table to ensure that the initial object position is kept constant.
						\end{itemize}
					\subsubsection{Software and Libraries used}
						\begin{itemize}
							\item KUKA youBot drivers.
							\item Control scripts for the arm to pick and move the objects in one of the three predefined placing poses.
							\item Marker pose subscribed script, to gather pose of the object.
							\item \textbf{LibreOffice Calc} for data management.
							\item \textbf{Python} for data visualization and calculations
							\item \textcolor{red}{Python libraries}:
							%						\begin{itemize}
							%							\item pandas
							%							\item numpy
							%							\item matplotlib
							%							\item seaborn
							%							\item scipy.stats
							%						\end{itemize}
							\end{itemize}		
											
				\subsection{Procedure}	
					\begin{itemize}
						\item First we run the script to get the arm in the pre-grasp position.
						\item We then place the object in the container, keeping the marker's orientation constant throughout the experiment.\\
						\item We then run the script to move the arm in one of the three pre-defined positions; and repeat this twenty times for each weight and pose combination. Thus, giving us 180 readings of pose coming from three different objects in three different orientations.
						\item Once the object is placed and the arm moves back to a stationary position, we run the subscriber script to collect pose readings of 50 frames from the camera. This is repeated after each motion.
					\end{itemize}
				\subsection{Expected Problems and Performance}
					\begin{itemize}
						\item The picking position of the arm might differ from the ground truth, because of vibrations, motion in the table and a variety of other physical conditions.
						\item The placement of the object in the container might not always be aligned properly.
						\item The marker on top of the object might move during movement and thus will lead to improper pose data.
						\item After placing the object, the gripper might touch it while moving away, which will introduce distortions in data.
						\item The light might not always be uniform, which might also cause some distortions in observation.						
					\end{itemize}
					
				\section{\textcolor{red}{Observations and Data}}
					 \subsection{Visualization}
						 \subsubsection{Data Visualization}
							  \textcolor{red}{Histograms of raw data}
	    				 \subsubsection{Outlier Detection and Removal}
		    				 \textcolor{red}{How we remove outliers}
						 \subsubsection{Pose Visualization}
							 \textcolor{red}{Plot Pose for data after removing outliers}				
					
%				\section{\textcolor{blue}{Results}}
%				\subsection{Final Position \& Accuracy }
%				  \textcolor{red}{Display in a tabular format}
%				\subsection{Compare Data with Gaussian}
				\end{document}
