	\documentclass[10pt,a4paper]{article}
	\usepackage[english]{babel}
	\usepackage[utf8]{inputenc}
	\usepackage{fancyhdr}
	\usepackage{hyperref}
	\usepackage{graphicx}
	\usepackage{caption}
	\usepackage{cite}
	
	\pagestyle{fancy}
	\fancyhf{}
	\rhead{19-April-2018}
	\lhead{Assignment 01}
	\rfoot{Page \thepage}
	 
	\begin{document}
	\begin{titlepage}
	\centering
	
		{\scshape\LARGE Scientific Experimentation and Evaluation\par}
		
		{\scshape\Large Assignment: 01\par}
	
		\vfill
		
		\vfill
		{\Large\itshape Anees Khan (9030423)
			\\Debaraj Barua (9030412)\\
			Md Zahiduzzaman (9030432)
			\par}
		\vfill
			
		{\large 19-April-2018\par}
	\end{titlepage}
	\tableofcontents
	
	%###########################################################################################%
	\newpage
	\section{Task 1: Formalization of general terms}
		\paragraph{Measured value:}
			\begin{itemize}
				\item Force using NI DAQ
				\item Velocity Using EPOS2
				\item Motor Position using EPOS2
			\end{itemize}
		\paragraph{Measurement result:}
			\begin{itemize}
				\item Force measured by sensor for a chosen depth between 0 and 11.5 mm. 
			\end{itemize}
		\paragraph{DUT:}
			\begin{itemize}
				\item Simulab complex tissue model.
			\end{itemize}
		\paragraph{Measuring facility:}
			\begin{itemize}
				\item A 10 DOF (Degrees of freedom) robotic system consisting of Physik instrumente Hexapod and monocarrier drive, which is connected with an indenter. 
				\item Force torque sensor is attached to indenter to provide axial force measurement.     
				\item Force measurements are read using NI DAQ. EPOS2 motor controller is used to read position and velocity of motors. 
			\end{itemize}
		\paragraph{Measuring System:}
			\begin{itemize}
				\item A 10 DOF (Degrees of freedom) robotic system consisting of Physik instrumente Hexapod and monocarrier drive, which is connected with an indenter. 
				\item Force torque sensor is attached to indenter to provide axial force measurement.     
				\item Force measurements are read using NI DAQ. EPOS2 motor controller is used to read position and velocity of motors. 
				\item Simulab complex tissue model (DUT).
			\end{itemize}
		\paragraph{Measuring principle:}
			\begin{itemize}
				\item Measuring force by moving indenter to different positions with a chosen depth between 0 and 11.5 mm.
			\end{itemize}
		\paragraph{Measuring Method:}
			\begin{itemize}
			\item To measuring force by virtue of which we determine if the surface of tissue under consideration has a uniform or non-uniform surface.
			\end{itemize}
		\paragraph{Sensitivity:}
			\begin{itemize}
			\item 0.04 N 
			\end{itemize}
	%###########################################################################################%
	\section{Task 2: Design of Experiment}
	\subsection{Design of Robot}
		\begin{itemize}
			\item The robot has been designed with three wheels.
			\item Two of these are driving wheels and are connected to the motors, thus enabling a differential drive systems; and the third is a driven wheel.
			\item Two pens will be fixed near the two driving wheels. The lines joining these two points will be in parallel to the driving axle (considering the axis between two driving wheels).
			\item The axis formed between these two points will be used to mark the orientation of robot, with respect to the coordinate system defined (described below).
			\item The starting position will be when this line lies on the x-axis of the coordinate system; the end position can be relatively measured.
		\end{itemize}
	
	\subsection{Measurement Process}
		\begin{itemize}
			\item \textbf{Measured Value}: Observable pose variation for three different constant velocity motions: 
				\begin{itemize}
					\item An arc to the left
					\item A straight line ahead
					\item An arc to the right
				\end{itemize}
			 \item \textbf{Experiments}: 
				 \begin{itemize}
				 	\item Constant angular and translational speed for a fixed time period to describe an arc to left.
				 	\item Constant translational speed and no angular speed for a fixed time period to describe a straight line.
				 	\item Constant angular and translational speed for a fixed time period to describe an arc to right.
				 \end{itemize}
			\item \textbf{Measurement Procedure} 
				\begin{itemize}
					\item Length and angles will be measured manually from the start and stop position.
					\item The positions are marked on a grid paper, on which the robot moves.
					\item A coordinate system is defined on the paper for reference and calculation of distances and angles.
					\item The start position can be marked such that one of the fixed pen lies over the origin of said coordinate system and the other pen lies on X-axis. 
					\item This will enable a fixed starting point for all the experiments.
				\end{itemize}
			 
		\end{itemize}
	\subsection{Expected Problems and Performance}
		\begin{itemize}
			\item Axis connecting the two pens might not be parallel to the wheel axle.	
			\item Start position of each run may not be exactly similar owing to inaccurate positioning of the robot, this will result in lower precision.
			\item Pens may slip of move during the run, as such may not result in accurate positions, thus affecting the precision of our readings.
			\item The constant angular and translational speeds that we assume, may be inaccurate. The actual speed may differ and thus our estimate from the time will be inaccurate.
			\item The initial acceleration and final deceleration of the robot has not been considered in the experiments, resulting in low accuracy.					
			\item In addition of the two previous points, slippage in the wheels and motors will also affect the accuracy of readings.
		\end{itemize}

	\end{document}